
\documentclass[a4paper,11pt]{article}

\usepackage[english]{babel}
\usepackage[latin1]{inputenc}
\usepackage[T1]{fontenc}
\usepackage{amsmath}
\usepackage{varwidth,xcolor}
\usepackage{graphicx}
\usepackage{amssymb}


\newcommand{\veps}{\varepsilon}
\newcommand{\ee}{{\rm e}\hspace{1pt}}


\begin{document}
\pagestyle{empty}


  \begin{center}
{\large {\bf 
Review on a report \\
``Model-free deep reinforcement learning'' \\
by Anniina Kinnunen} \\

\bigskip

Written by: 

Mikael Fredriksson
 } 
\end{center}

As the paper is currently as short as it is, it will be quite difficult to give 
adequate feedback. I'll try my best, though.

The structure seems sound when looking at the currently existing text and the planned
additions. The text is quite clear and visuals and the one equation help in understanding.
Thank you for explaining thoroughly what everything means in the equation.

The basic description of reinforcement learning is clear. You could maybe consider
whether adding some equations would make it even clearer but it's already good as it is.

Additionally, I listed some ideas for possible corrections below:

\begin{itemize}
	
	\item Claiming that there are exactly three categories of machine learning 
		  is somewhat questionable in my opinion. I would personally use a more
		  vague statement such as "reinforcement learning is one of the categories of
		  machine learning, which include e.g. unsupervised and supervised learning".
		  Depending on the source, you could argue semi-supervised learning being its
		  own category. I even saw some claiming deep learning being its own category.
		  Generally I usually try to avoid such sweeping statements about large
		  fields like machine learning.

\end{itemize}


\end{document}
